\documentclass[a4paper, 12pt, twoside]{article}
\usepackage{fancyhdr}
\usepackage{amsmath}
\usepackage{graphicx}

\pagestyle{fancy}
\fancyhf{}
\rhead{Joseph Free \\ MATH 3261 \\ Project 5}
\lhead{QR Decomposition and Least Squares Solutions}
\addtolength{\headheight}{41.7pt}
\cfoot{\thepage}

\begin{document}
\section{Problem}
In this assignment we will be obtaining the least squares solution to the system

\begin{equation}
\label{one}
\begin{array}{rrrrr}
3x& - & 2y &= & 3 \\
  &   & 2y &= & 5\\
4x& + & 4y &= & 4\\ 
\end{array}
\end{equation}
by using $QR$ decomposition. The assignment will be carried out in four parts. In the first and second parts, we will use the Gram-Schmidt process and Householder transformations, respectively, to find the $QR$ decomposition of the coefficient matrix of the system above. In the third part, we will use a matlab command to generate the QR decomposition. The fourth part will contain the solution of (\ref{one}).

\subsection{Part 1: Gram-Schmidt}
Recall that a $QR$ decomposition of an $n \times k$ matrix $A$ is a factorization of A into the product $QR$, where $Q$ is an $n \times k$ orthogonal matrix ($Q^TQ = I$) and $R$ is a $k \times \ k$ upper triangular matrix.
Using the Gram-Schmidt process allows us to determine the columns of the matrix $Q$ by using the columns of $A$ as the set of vectors to orthonormalize. After obtaining Q, we can use the identity 
\begin{equation}
R = Q^TA
\end{equation}
to compute the individual components of R.

Proceeding with the Gram-Schmidt process using vectors $a_1 = (3,0,4)^t$,
$a_2 = (-2, 3, 4)^t$ from
$$
A=
\begin{bmatrix}
	3 & -2\\ 0 & 3\\ 4 & 4\\
\end{bmatrix}
$$
we obtain the following orthonormal vectors

$$ u_1 = \frac{a_1}{\lvert a_1 \rvert} = (\frac{3}{5},0,\frac{4}{5})^t$$
$$u_2 = \frac{w_2}{\lvert w_2 \rvert} = (\frac{-16}{25}, \frac{3}{5}, \frac{12}{25})^t$$
where $w_2 = {a_1} - (a_2u_1)u_1 = (-16/5, 3, 12/5)^t$. Using these vectors as the columns of $Q$, we have
$$
Q = 
\begin{bmatrix}
	3/5& & &-16/25\\
	0& & &	3/5\\
	4/5& & & 12/25\\	
\end{bmatrix}.
$$

Then, since A is $3 \times 2$, we know that $R$ must be $2 \times 2$. So by using (2), we obtain

$$
R = 
\begin{bmatrix}
	5 & 2\\
	0 & 5\\
\end{bmatrix}
$$.

We conclude that 
$$
A
=
\begin{bmatrix}
3/5& & &-16/25\\
0& & &	3/5\\
4/5& & & 12/25\\	
\end{bmatrix}.
\begin{bmatrix}
5 & 2\\
0 & 5\\
\end{bmatrix}
= QR
$$

\subsection{Part 2: Householder Transformations}

Using A as defined in part 1, we wish to apply orthogonal transformations $H_1, H_2$ to A in order to zero the entries below the main diagonal. If we take $Q = (H_2H_1)^T$, then we see that the resultant product $H_2H_1A = R$, as $A=QR$, hence $Q^TA = Q^TQR = R.$ 

Using the definition of the Householder reflector, we take $H_1 = I - 2u_1u_1^t$ where $u_1 = a_1 + sgn(a_1)\lVert a_1 \rVert e$ and $e = (1,0,0)^t$. Thus,

$$
u_1 = \frac{(3,0,4)^t + 5(1,0,0)^t}{\lVert (3,0,2)^t + 5(1,0,0)^t \rVert}
= \frac{1}{\sqrt{5}}(2,0,1)^t
$$

And

$$
H_1 = \frac{1}{5}
\begin{bmatrix}
-3 & 0 & -4 \\
0 & 0 & 0 \\
-4 & 0 & 3 \\
\end{bmatrix}
$$

Multiplying by A, we get

$$
H_1A =
\begin{bmatrix}
-5 & -2 \\ 0 & 3 \\ 0 & 4 \\
\end{bmatrix}
$$

Proceeding in a similar manner using the vector $a_2 = (3,4)^t$ from the second column of $H_1A$, we construct $u_2 = \frac{1}{\sqrt{5}} (2,1)^t$, from which we find $H_2$ to be

$$
H_2 = \frac{1}{5}
\begin{bmatrix}
-3 & -4 \\
-4 & 3 \\
\end{bmatrix}
$$ 

Multiplying $H_2$ by the block (vector, really) containing $a_2$, we obtain

$$
H_2A_{2:3,2} =
\frac{1}{5}
\begin{bmatrix}
-3 & -4 \\
-4 & 3 \\
\end{bmatrix}
\begin{pmatrix}
3 \\ 4
\end{pmatrix}
=
\begin{pmatrix}
-5 \\ 0
\end{pmatrix}
$$

Putting this all together, we may conclude that the upper triangular matrix R is given by

$$
R = 
\begin{bmatrix}
-5 & -2 \\
0 & -5 \\
0 & 0 \\
\end{bmatrix}.
$$

We may then construct Q as the product of $H_1$ and $\tilde{H}_2$, where 

$$
\tilde{H}_2 = \begin{bmatrix} 1 & 0 \\ 0 & H_2\\ \end{bmatrix}
$$

Thus, $Q$ is given by

$$
Q = 
\frac{1}{5}
\begin{bmatrix}
-3 & 0 & -4 \\
0 & 0 & 0 \\
-4 & 0 & 3 \\
\end{bmatrix}
\begin{bmatrix}
1 & 0 & 0 \\
0 &-3/5 & -4/5 \\
0 &-4/5 & 3/5 \\
\end{bmatrix}
=
\begin{bmatrix}
-3/5 & 16/25 & -12/25 \\
0 &0 &0 \\
-4/5& -12/25 & 9/25 \\
\end{bmatrix}.
$$

\pagebreak

\subsection*{Part 3: MATLAB}

Using the qr function in MATLAB, the following results are generated.

\begin{verbatim}
>> A = [3 -2; 0 3; 4 4] 

A =

3    -2
0     3
4     4

>> [Q, R] = qr(A)

Q =

-0.6000    0.6400   -0.4800
0   -0.6000   -0.8000
-0.8000   -0.4800    0.3600

R =

-5    -2
0    -5
0     0

\end{verbatim}

\subsection{Part 4: Least Squares Solution}

Using the Gram-Schmidt QR decomposition, the solution to the system can be found by solving 

$$
\begin{bmatrix}
5 & 2 \\ 0 & 5\\
\end{bmatrix}
\begin{bmatrix} x \\ y \end{bmatrix}
= 
\begin{bmatrix}
3/5& & &-16/25\\
0 & & & 3/5\\
4/5& & & 12/25\\	
\end{bmatrix}^T
\begin{bmatrix} 3 \\ 5 \\ 4 \end{bmatrix}
$$

By using any conventional method, we find the solution to be

$$
\begin{bmatrix} x \\ y \end{bmatrix}
=
\begin{bmatrix} 19/25 \\ 3/5 \end{bmatrix}
$$

\end{document}
